% !TeX spellcheck = de_DE_frami
%%%%%%%%%%%%%%%%%%%%%%%%%%%%%%%%%%%%%%
%%%%%%%%%%%%%%%%%%%%%%%%%%%%%%%%%%%%%%
\section{Hinweise zu kommenden Wochen}

\begin{frame}{Hinweise zur Klausur}
	\underline{Anmeldung zur 1. Klausur}:\\\smallskip
	\textbf{Per Email} (an \url{mda.klausur@cs.tu-dortmund.de}) unter Angabe von\\
	
	\begin{itemize}
		\item Name, Vorname, Matrikelnummer, Studiengang\\
	\end{itemize}
	
	Anmeldeschluss: 14 Tage vor dem Klausurtermin.\\
	\bigskip
	
	\underline{Klausurtermin}:\\\smallskip
	Montag, \textbf{27.02.2023, 9:00--10:30\,Uhr}, Otto-Hahn-Str.\,14, E23.\\
	\medskip
	
	Der 2. Klausurtermin ist für Ende März geplant.\\
\end{frame}

\begin{frame}{Hinweise zur Vorlesung nächste Woche}
	Prof. Meinard Müller vom \emph{International Audio Laboratories Erlangen}\\
	\hfill wird per Zoom in den Vorlesungsraum (SRG1, R1.004) geschaltet.\\
	\bigskip
	
	\centering
	\underline{Thema}\\\smallskip
	\textbf{FMP Notebooks: Interaktives Lehren\\
		und
	Lernen der Digitalen Musikverarbeitung}
	
	\bigskip\bigskip
	\emph{Ich freue mich über euer Erscheinen!} 
\end{frame}

\begin{frame}{Hinweise zur Übung nächste Woche}
Geplant als praktisches \textbf{Hands-On} zur Thematik.\\
\medskip

Empfehlung: \textbf{Ausprobieren} von \textbf{Python} mit \textbf{Scamp}-Library\\

{\small\itshape Aber: Aufgaben sind so gestaltet, dass sie generell auch in R und Matlab lösbar sind.}\\
\medskip

Übungsaufgaben finden sich ab heute Abend (27.1.) im \emph{Moodle}.
\bigskip

Detail-Fragen zu den Inhalten/Folien von heute sind willkommen!
\end{frame}