% !TeX root = MDA_VL_Ostermann_Folien.tex
\usepackage[utf8]{inputenc}
\usepackage[english, german, ngerman]{babel}
\usepackage[T1]{fontenc}
\usepackage{lmodern}

\usepackage{graphicx}
\usepackage{xcolor}
\usepackage[normalem]{ulem}

\usepackage{hyperref}
\usepackage[autostyle=true,german=quotes]{csquotes}
\usepackage[square, numbers]{natbib}%numbers
\let\oldcitep\citep
\renewcommand{\citep}[2][]{$^\text{\oldcitep[#1]{#2}}$}

\usepackage[]{todonotes}
\newcommand{\missingfig}[2][]{\scalebox{.5}{\missingfigure[figwidth=7cm,figheight=3.5cm, #1]{#2}}}
\newcommand{\todobox}[2][orange]{\todo[inline,color=#1]{#2}}

\usepackage{caption}
\usepackage{subcaption}
\usepackage{booktabs}
\usepackage{tabularx}
\usepackage{ragged2e}
\usepackage{multirow}
\usepackage{csvsimple} % to load csv files

%%%%% Infoboxen %%%%%
\usepackage{mdframed}
\newenvironment{infobox}[2][\textwidth]{
% [#1]=width, {#2}=titel
	\mdfsetup{%
		frametitle={%
			\tikz[baseline=(current bounding box.east),outer sep=0pt]
			\node[anchor=east,rectangle,fill=TUDoGreen,text=white]
			{#2};
		},
		%innertopmargin=10pt,
		linecolor=TUDoGreen,%
		linewidth=2pt,%topline=true,%
		frametitleaboveskip=-8pt,%
	}
	\begin{minipage}{#1}
	\begin{mdframed}[]\relax%
	}{
	\end{mdframed}
	\end{minipage}
}

%%%%% SI-Einheiten %%%%%%
\usepackage[detect-weight]{siunitx}%decimalsymbol=comma

\usepackage{romannum}

\usepackage{amssymb}
\usepackage{marvosym}
\usepackage{wasysym}
%\usepackage{MnSymbol}
%\usepackage{physics} % vector with \va \vb or \va* \vb*

\usepackage{amssymb}
\usepackage{multimedia}

%%%%%% simple audio player that triggers default media player (e.g. vlc)
% former on symbol: \raisebox{1pt}{$\blacktriangleright$}
% former off symbol: $\blacksquare$
\newcommand{\audioplayer}[1]{\mbox{\sound[externalviewer]{\faVolumeUp}{#1}~\sound[externalviewer]{\faVolumeOff}{audio/stop.mp3}}}


%\definecolor{TUDoGreen}{RGB}{132,184,24} % real tu green
\definecolor{TUDoGreen}{RGB}{110,165,20} % darker tu green
\colorlet{BoxCol}{TUDoGreen}

%%%% Notes in Beamer:
%https://tex.stackexchange.com/questions/48402/personal-notes-when-preparing-a-talk-with-latex-beamer-class
\ifshownotes
	\setbeamertemplate{note page}[plain]% default, plain, compress
	\usepackage{pgfpages}
	%\setbeameroption{show notes}
	\setbeameroption{show notes on second screen=right}
	\let\notebak\note
	\renewcommand{\note}[1]{\notebak{
	\begin{frame}\begin{quote}\sf\scriptsize #1 \addtocounter{framenumber}{-1}\end{quote}\end{frame}
	}}
\fi


\title{\vspace{-2cm}\newline\includegraphics[height=0.08\textheight]{logos/TU_Dortmund.jpg}\hspace{0.33\textwidth}\includegraphics[height=0.08\textheight]{logos/Logo_50-Jahre-Informatik.pdf}\vspace{0.5cm}\newline\thetitle}
\author{\theauthor}
% insert correct institute name
%\email{fabian.ostermann@tu-dortmund.de}
\date{\small\thedate}
\institute{\theinstitute}
\subtitle{\vspace{0.25cm}\thesubtitle\vspace{-0.25cm}}
%\titlegraphic{}

\setbeamercolor{background canvas}{bg=white}
\setbeamercovered{invisible}
\usecolortheme[named=TUDoGreen]{structure}
\setbeamertemplate{frametitle}[default][center]
\setbeamerfont{frametitle}{size=\Large,series=\bfseries}
\setbeamerfont{framesubtitle}{size=\normalsize,series=\bfseries}

\usepackage{multicol}

%%%%%% draw at for drawing objects at specific positions
% https://tex.stackexchange.com/questions/82463/how-to-position-a-beamer-box-in-a-slide
\usepackage[absolute,overlay]{textpos}
\newcommand{\drawat}[2]{ % #1=x,y, #2=content
	\begin{textblock*}{0mm}(#1)
		#2
	\end{textblock*}
}
%%% the grid %%%
\usepackage[texcoord,
%grid,
gridunit=mm,gridcolor=black!10,subgridcolor=black!5,gridBG=false]{eso-pic}

%%%% make a file that lists all frametitles
\newwrite\slideinfofile
\immediate\openout\slideinfofile=slideinfo.txt
%\newcounter{SlideNumber}
\AtBeginEnvironment{frame}{
		\immediate\write\slideinfofile{%\theSlideNumber
		[\insertframenumber] \insertframetitle}
}

%%%%% Kommando für automatisch einblendendes Inhaltsverzeichnis am Anfang jeder Section
\newcommand{\toc}[1][currentsection]{
	\begin{frame}{Übersicht}
		\only<1>{\tableofcontents[]}
		\only<2->{\tableofcontents[#1]}
	\end{frame}
}
\AtBeginSection[]{\toc}

%%% Useful Symbols %%%%
\usepackage{fontawesome}
\newcommand{\warnSign}{{\color{TUDoGreen}\faWarning}}
\newcommand{\conclude}{{\color{TUDoGreen}$\boldsymbol{\rightarrow}$}}
\newcommand{\Conclude}{{\color{TUDoGreen}$\boldsymbol{\Rightarrow}$}}
\newcommand{\questSign}{{\color{TUDoGreen}\fbox{\textbf{?}}}}
\newcommand{\attentionSign}{{\color{TUDoGreen}\fbox{\textbf{!}}}}
\newcommand{\ideaSign}{{\color{TUDoGreen}\fbox{\textbf{\faLightbulbO}}}}

\usepackage{musicography}
\renewcommand{\musEighth}{\musFlaggedNote{\symbol{7}}{\symbol{40}}}
%\musHalf already working

